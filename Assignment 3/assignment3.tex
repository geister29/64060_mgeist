% Options for packages loaded elsewhere
\PassOptionsToPackage{unicode}{hyperref}
\PassOptionsToPackage{hyphens}{url}
%
\documentclass[
]{article}
\usepackage{lmodern}
\usepackage{amsmath}
\usepackage{ifxetex,ifluatex}
\ifnum 0\ifxetex 1\fi\ifluatex 1\fi=0 % if pdftex
  \usepackage[T1]{fontenc}
  \usepackage[utf8]{inputenc}
  \usepackage{textcomp} % provide euro and other symbols
  \usepackage{amssymb}
\else % if luatex or xetex
  \usepackage{unicode-math}
  \defaultfontfeatures{Scale=MatchLowercase}
  \defaultfontfeatures[\rmfamily]{Ligatures=TeX,Scale=1}
\fi
% Use upquote if available, for straight quotes in verbatim environments
\IfFileExists{upquote.sty}{\usepackage{upquote}}{}
\IfFileExists{microtype.sty}{% use microtype if available
  \usepackage[]{microtype}
  \UseMicrotypeSet[protrusion]{basicmath} % disable protrusion for tt fonts
}{}
\makeatletter
\@ifundefined{KOMAClassName}{% if non-KOMA class
  \IfFileExists{parskip.sty}{%
    \usepackage{parskip}
  }{% else
    \setlength{\parindent}{0pt}
    \setlength{\parskip}{6pt plus 2pt minus 1pt}}
}{% if KOMA class
  \KOMAoptions{parskip=half}}
\makeatother
\usepackage{xcolor}
\IfFileExists{xurl.sty}{\usepackage{xurl}}{} % add URL line breaks if available
\IfFileExists{bookmark.sty}{\usepackage{bookmark}}{\usepackage{hyperref}}
\hypersetup{
  hidelinks,
  pdfcreator={LaTeX via pandoc}}
\urlstyle{same} % disable monospaced font for URLs
\usepackage[margin=1in]{geometry}
\usepackage{graphicx}
\makeatletter
\def\maxwidth{\ifdim\Gin@nat@width>\linewidth\linewidth\else\Gin@nat@width\fi}
\def\maxheight{\ifdim\Gin@nat@height>\textheight\textheight\else\Gin@nat@height\fi}
\makeatother
% Scale images if necessary, so that they will not overflow the page
% margins by default, and it is still possible to overwrite the defaults
% using explicit options in \includegraphics[width, height, ...]{}
\setkeys{Gin}{width=\maxwidth,height=\maxheight,keepaspectratio}
% Set default figure placement to htbp
\makeatletter
\def\fps@figure{htbp}
\makeatother
\setlength{\emergencystretch}{3em} % prevent overfull lines
\providecommand{\tightlist}{%
  \setlength{\itemsep}{0pt}\setlength{\parskip}{0pt}}
\setcounter{secnumdepth}{-\maxdimen} % remove section numbering
\ifluatex
  \usepackage{selnolig}  % disable illegal ligatures
\fi

\author{}
\date{\vspace{-2.5em}}

\begin{document}

\hypertarget{partition-the-data-into-training-60-and-validation-40-sets.a.create-a-pivot-table-for}{%
\section{Partition the data into training (60\%) and validation (40\%)
sets.A.Create a pivot table
for}\label{partition-the-data-into-training-60-and-validation-40-sets.a.create-a-pivot-table-for}}

\hypertarget{the-training-data-with-online-as-a-column-variable-cc-as-a-row-variable-and-loan-as-a}{%
\section{the training data with Online as a column variable, CC as a row
variable, and Loan as
a}\label{the-training-data-with-online-as-a-column-variable-cc-as-a-row-variable-and-loan-as-a}}

\hypertarget{secondary-row-variable.-the-values-inside-the-table-should-convey-the-count.-in-r-use}{%
\section{secondary row variable. The values inside the table should
convey the count. In R
use}\label{secondary-row-variable.-the-values-inside-the-table-should-convey-the-count.-in-r-use}}

\hypertarget{functions-meltand-cast-or-function-table.-in-python-use-panda-dataframe-methods}{%
\section{functions melt()and cast(), or function table(). In Python, use
panda dataframe
methods}\label{functions-meltand-cast-or-function-table.-in-python-use-panda-dataframe-methods}}

\hypertarget{meltand-pivot.b.consider-the-task-of-classifying-a-customer-who-owns-a-bank-credit-card}{%
\section{melt()and pivot().B.Consider the task of classifying a customer
who owns a bank credit
card}\label{meltand-pivot.b.consider-the-task-of-classifying-a-customer-who-owns-a-bank-credit-card}}

\hypertarget{and-is-actively-using-online-banking-services.-looking-at-the-pivot-table-what-is-the}{%
\section{and is actively using online banking services. Looking at the
pivot table, what is
the}\label{and-is-actively-using-online-banking-services.-looking-at-the-pivot-table-what-is-the}}

\hypertarget{probability-that-this-customer-will-accept-the-loan-offer-this-is-the-probability-of}{%
\section{probability that this customer will accept the loan offer?
{[}This is the probability
of}\label{probability-that-this-customer-will-accept-the-loan-offer-this-is-the-probability-of}}

\hypertarget{loan-acceptance-loan-1-conditional-on-having-a-bank-credit-card-cc-1-and-being-an}{%
\section{loan acceptance (Loan = 1) conditional on having a bank credit
card (CC = 1) and being
an}\label{loan-acceptance-loan-1-conditional-on-having-a-bank-credit-card-cc-1-and-being-an}}

\hypertarget{active-user-of-online-banking-services-online-1.c.create-two-separate-pivot-tables-for}{%
\section{active user of online banking services (Online = 1){]}.C.Create
two separate pivot tables
for}\label{active-user-of-online-banking-services-online-1.c.create-two-separate-pivot-tables-for}}

\hypertarget{the-training-data.-onewill-have-loan-rows-as-a-function-of-online-columns-and-the-other}{%
\section{the training data. Onewill have Loan (rows) as a function of
Online (columns) and the
other}\label{the-training-data.-onewill-have-loan-rows-as-a-function-of-online-columns-and-the-other}}

\hypertarget{will-have-loan-rows-as-a-function-of-cc.d.compute-the-following-quantities-pa-b-means}{%
\section{will have Loan (rows) as a function of CC.D.Compute the
following quantities {[}P(A \textbar{} B)
means}\label{will-have-loan-rows-as-a-function-of-cc.d.compute-the-following-quantities-pa-b-means}}

\hypertarget{the-probability-ofa-given-b-i.pcc-1-loan-1-the-proportion-of-credit-card-holders}{%
\section{``the probability ofA given B''{]}: i.P(CC = 1 \textbar{} Loan
= 1) (the proportion of credit card
holders}\label{the-probability-ofa-given-b-i.pcc-1-loan-1-the-proportion-of-credit-card-holders}}

\hypertarget{among-the-loan-acceptorsii.ponline-1-loan-1-iii.ploan-1-the-proportion-of}{%
\section{among the loan acceptors)ii.P(Online = 1 \textbar{} Loan = 1)
iii.P(Loan = 1) (the proportion
of}\label{among-the-loan-acceptorsii.ponline-1-loan-1-iii.ploan-1-the-proportion-of}}

\hypertarget{loan-acceptors-iv.pcc-1-loan-0-v.ponline-1-loan-0vi.ploan-0e.use-the}{%
\section{loan acceptors) iv.P(CC = 1 \textbar{} Loan = 0) v.P(Online = 1
\textbar{} Loan = 0)vi.P(Loan = 0)E.Use
the}\label{loan-acceptors-iv.pcc-1-loan-0-v.ponline-1-loan-0vi.ploan-0e.use-the}}

\hypertarget{quantities-computed-above-to-compute-the-naive-bayes-probability-ploan-1-cc-1}{%
\section{quantities computed above to compute the naive Bayes
probability P(Loan = 1 \textbar{} CC =
1,}\label{quantities-computed-above-to-compute-the-naive-bayes-probability-ploan-1-cc-1}}

\hypertarget{online-1.f.compare-this-value-with-the-one-obtained-from-the-pivot-table-in-b.}{%
\section{Online = 1).F.Compare this value with the one obtained from the
pivot table in
(B).}\label{online-1.f.compare-this-value-with-the-one-obtained-from-the-pivot-table-in-b.}}

\hypertarget{which-is-a-more-accurate-estimateg.which-of-the-entries-in-this-table-are-needed-for}{%
\section{Which is a more accurate estimate?G.Which of the entries in
this table are needed
for}\label{which-is-a-more-accurate-estimateg.which-of-the-entries-in-this-table-are-needed-for}}

\hypertarget{computing-ploan-1-cc-1-online-1-run-naive-bayes-on-the-data.-examine-the-model}{%
\section{computing P(Loan = 1 \textbar{} CC = 1, Online = 1)? Run naive
Bayes on the data. Examine the
model}\label{computing-ploan-1-cc-1-online-1-run-naive-bayes-on-the-data.-examine-the-model}}

\hypertarget{output-on-training-data-and-find-the-entry-that-corresponds-to-ploan-1-cc-1}{%
\section{output on training data, and find the entry that corresponds to
P(Loan = 1 \textbar{} CC =
1,}\label{output-on-training-data-and-find-the-entry-that-corresponds-to-ploan-1-cc-1}}

\hypertarget{online-1.-compare-this-to-the-number-you-obtained-in-e.}{%
\section{Online = 1). Compare this to the number you obtained in
(E).}\label{online-1.-compare-this-to-the-number-you-obtained-in-e.}}

\hypertarget{load-everything-from-before}{%
\subsection{load everything from
before}\label{load-everything-from-before}}

library(readr) library(caret) library(reshape2) library(ISLR)
library(e1071) library(tidyverse) library(pivottabler) library(class)
library(gmodels) library(ggplot2) library(dplyr) library(tidyr)
library(magrittr) library(fastDummies) library(rmarkdown) \#\# load the
file UniversalBank \textless-
read\_csv(``GitHub/64060\_mgeist/Assignment 3/UniversalBank.csv'')
UniversalBank\(PersonalLoan = as.factor(UniversalBank\)PersonalLoan)
UniversalBank\(Online = as.factor(UniversalBank\)Online)
UniversalBank\(CreditCard = as.factor(UniversalBank\)CreditCard)
View(UniversalBank) set.seed(123)

\hypertarget{partitiion-data-6040}{%
\subsection{partitiion data 60/40}\label{partitiion-data-6040}}

Index\_Train \textless- sample(row.names(UniversalBank),
0.6*dim(UniversalBank){[}1{]})\\
Index\_Test \textless- setdiff(row.names(UniversalBank), Index\_Train)

\#\#Part A print(``Part A'') partA \textless- UB\_Train
table(``CC''=partA\(CreditCard,"PL"=partA\)PersonalLoan,``O/L''=partA\(Online) table1 = recast(partA,partA\)CreditCard+partA\(PersonalLoan~partA\)Online)
table1

\#\#Part B print(``Part B'') print(``The probability of the customer
accepting the loan is 82/882 or 9.2\%'')

\#\#Part C print(``Part C'') table2 =
recast(partA,partA\(PersonalLoan~partA\)Online) table2 table3 =
recast(partA,partA\(CreditCard~partA\)Online) table3

\#\#Part D print(``Part D'') print(``CC=credit card, PL=personal loan,
OL= online account'') print(``Part 1 is P(CC\textbar PL) = 77/(77+198)
is 28.00\%'') print(``Part 2 is P(OL\textbar PL) = 166/(166+109) is
60.30\%'') print(``Part 3 is P(PL) = 275/(275+2725) is 10.09\%'')
print(``Part 4 is P(CC\textbar PL') = 801/(801+1924) is 29.39\%'')
print(``Part 5 is P(OL\textbar PL') = 1588/(1588+1137) is 58.27\%'')
print(``Part 6 is P(PL') = 2725/(275+2725) is 90.83\%'')

\#\#Part E print(``Part E'')
print(``P(CC\textbar PL)\emph{P(OL\textbar PL)}P(PL)'')
print(``\_\_\_\_\_\_\_\_\_\_\_\_\_\_\_\_\_\_\_\_\_\_\_'')
print("P(CC\textbar PL)\emph{P(OL\textbar PL)P(PL)``) print(''+``)
print(''P(CC\textbar PL')}P(OL\textbar PL')\emph{P(PL')")
((77/(77+198))}(166/(166+109))\emph{(275/(275+2725)))/(((77/(77+198))}(166/(166+109))\emph{(275/(275+2725)))+((801/(801+1924))}(1588/(1588+1137))*2725/(2725+275)))

\#\#Part F print(``Part F'') print(``Part B = 9.29\% and Part E =
9.05\%'') print(``They are very similar. I believe that we the most
accurate one is part be because it was an actual calculation, in this
instance Naive Bayes is not needed.'')

\#\#Part H print(``Part H'')

train.df \textless- UniversalBank{[}Index\_Train, {]} test.df \textless-
UniversalBank{[}Index\_Test, {]} train \textless-
UniversalBank{[}Index\_Train, {]} test = UniversalBank{[}Index\_Test,{]}

nb\_train = train.df{[},c(10,13:14){]} nb\_test =
test.df{[},c(10,13:14){]} naivebayes =
naiveBayes(PersonalLoan\textasciitilde.,data=nb\_train) naivebayes

\end{document}
